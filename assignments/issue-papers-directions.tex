\hypertarget{paper-guidelines}{%
\subsection{Paper Guidelines}\label{paper-guidelines}}

Each issue paper will focus on a different aspect of the one problem or
policy issue you choose. \textbf{I recommend, but won't require you to
pick a national issue}. You should include at least four \emph{academic}
sources (not including course readings) for each paper. Appropriate
sources include (for example, but not limited to), peer-reviewed journal
articles, books, government reports, research institute (``think tank'')
reports, and mainstream newspaper articles.

\vspace{0.1in}

\noindent I recommend making use of the
\href{https://www.comparativeagendas.net/}{Comparative Agendas Project}
website. Be sure and use the United States by going to ``Select a
Project'' on the right-hand side of the screen and choose the United
States.

\hypertarget{issue-paper-1-define-the-problem}{%
\subsubsection{Issue Paper 1: Define the
Problem}\label{issue-paper-1-define-the-problem}}

In issue paper 1, you should provide a brief overview of your policy
issue to introduce the reader to the issue. It can include answers to
such questions as, why is it an important issue? What makes it a
societal problem? Why should policy makers be concerned about the issue?
Define some of the key terms with regard to the issue. Make clear if you
are examining an issue on the local, state, or \textbf{federal} level.
\textbf{Due: Sept 15}

\hypertarget{issue-paper-2-institutions-subsystems-and-agenda-setting}{%
\subsubsection{Issue Paper 2: Institutions, Subsystems, and
Agenda-Setting}\label{issue-paper-2-institutions-subsystems-and-agenda-setting}}

In issue paper 2, describe the institutions involved in policymaking
with regard to your policy problem. What congressional committees,
executive branch agencies, and interest groups are involved in your
issue? Finally, what are some ways for your issue to reach the
policymaking agenda. \textbf{Due: Oct 6}

\hypertarget{issue-paper-3-current-policies}{%
\subsubsection{Issue Paper 3: Current
Policies}\label{issue-paper-3-current-policies}}

In issue paper 3, you should discuss what policy or policies are
currently in place to address your issue. Is there legislation? Who
implements it? Are there regulations? What are some of the current
policy tools that are being used with regard to the issue? \textbf{Due:
Nov 3}

\hypertarget{issue-paper-4-policy-alternatives}{%
\subsubsection{Issue Paper 4: Policy
Alternatives}\label{issue-paper-4-policy-alternatives}}

In issue paper 4, you should briefly describe three policy alternatives
that could be implemented to address your policy issue. In your
description, be sure and describe the pros and cons of each approach.
Then, make a recommendation about which policy alternative a
decision-maker should select and why. Finally, briefly describe how the
policy will be implemented. \textbf{Due: Nov 22}
