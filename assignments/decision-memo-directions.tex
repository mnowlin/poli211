\hypertarget{decision-memo-guidelines}{%
\subsection{Decision Memo Guidelines}\label{decision-memo-guidelines}}

Generally, a decision memo is a very short and extremely concise
document that summarizes a policy problem, briefly considers policy
options, and makes a recommendation to a decision maker. Your decision
memo will be based on the information in your issue papers.

\vspace{0.1in}
\noindent The decision memo should be two pages, single spaced, with no
references or footnotes. Use the Pennock chapter on
\href{https://lms.cofc.edu/d2l/home}{OAKS} as guidance.

\hypertarget{outline}{%
\subsection{Outline}\label{outline}}

Your decision memo should follow the outline below and contain each of
the following elements. \textbf{Use the following headings in your
memo}. I strongly recommended using the example outline in Pennock on pg
179.

\hypertarget{heading}{%
\subsubsection{Heading}\label{heading}}

You do not need to title this section. The decision memo must have this
heading:

\vspace{0.1in}
TO: \emph{Decision maker that is the intended audience}

FROM: \emph{Your Name}

SUBJECT: \emph{Policy issue you are addressing}

DATE: \emph{Due date of the assignment}

\hypertarget{executive-summary}{%
\subsubsection{Executive Summary}\label{executive-summary}}

A brief abstract outlining what's in your memo, including your policy
recommendation.

\hypertarget{background}{%
\subsubsection{Background}\label{background}}

A brief background about the problem. Use information from your first three issue papers. 

\hypertarget{alternatives}{%
\subsubsection{Alternatives}\label{alternatives}}

Consider three policy alternatives to address the problem. Take
information from your fourth issue paper.

\hypertarget{recommendation}{%
\subsubsection{Recommendation}\label{recommendation}}

Make a recommendation of one of the three policy alternatives. Draw from your fourth issue paper. 
