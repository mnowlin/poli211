\newpage

\section{Important Information}\label{important-information}

\subsection{Academic Integrity Statement}\label{cheating-or-plagiarism}

Lying, cheating, attempted cheating, and plagiarism are violations of our Honor Code that, when identified, are investigated. Each incident will be examined to determine the degree of deception involved.

\vspace{0.10in}
\noindent Incidents where the instructor determines the student’s actions are related more to a misunderstanding will handled by the instructor. A written intervention designed to help prevent the student from repeating the error will be given to the student. The intervention, submitted by form and signed both by the instructor and the student, will be forwarded to the Dean of Students and placed in the student’s file.

\vspace{0.10in}
\noindent Cases of suspected academic dishonesty will be reported directly by the instructor and/or others having knowledge of the incident to the Dean of Students. A student found responsible by the Honor Board for academic dishonesty will receive a XXF in the course, indicating failure of the course due to academic dishonesty. This status indicator will appear on the student’s transcript for two years after which the student may petition for the XX to be expunged. The F is permanent. 

\vspace{0.10in}
\noindent Students should be aware that unauthorized collaboration -- working together without permission -- is a form of cheating. Unless the instructor specifies that students can work together on an assignment, quiz and/or test, no collaboration during the completion of the assignment is permitted. Other forms of cheating include possessing or using an unauthorized study aid (which could include accessing information via a cell phone or computer), copying from others’ exams, fabricating data, and giving unauthorized assistance.

\vspace{0.10in}
\noindent Research conducted and/or papers written for other classes cannot be used in whole or in part for any assignment in this class without obtaining prior permission from the instructor.

\vspace{0.10in}
\noindent Students can find the complete Honor Code and all related processes in the \href{http://studentaffairs.cofc.edu/honor-system/studenthandbook/index.php}{Student Handbook} 

\subsection{Students with
Disabilities}\label{students-with-disabilities}

The College will make reasonable accommodations for persons with
documented disabilities. Students should apply at the
\href{http://disabilityservices.cofc.edu}{Center for Disability
Services} located on the first floor of the Lightsey Center, Suite 104.
Students approved for accommodations are responsible for notifying me as
soon as possible and for contacting me at least one week before any
accommodation is needed.

\subsection{Inclement Weather, Pandemic or Substantial Interruption of Instruction}

If in-person classes are suspended, faculty will announce to their students a detailed plan for a change in modality to ensure the continuity of learning. All students must have access to a computer equipped with a web camera, microphone, and Internet access. Resources are available to provide students with these essential tools.


\subsection{Mental and Physical Wellbeing}

At the college, we take every students’ mental and physical wellbeing seriously. If you find yourself experiencing physical illnesses, please reach out to student health services (843.953.5520). And if you find yourself experiencing any mental health challenges (for example, anxiety, depression, stressful life events, sleep deprivation, and/or loneliness/homesickness) please consider contacting either the Counseling Center (professional counselors at \href{http://counseling.cofc.edu} or 843.953.5640 3rd Robert Scott Small Building) or the Students 4 Support (certified volunteers through texting "4support" to 839863, visit \href{http://counseling.cofc.edu/cct/index.php}, or meet with them in person 3rd Floor Stern Center).  These services are there for you to help you cope with difficulties you may be experiencing and to maintain optimal physical and mental health.

\subsection{Food and Housing Resources}

Many CofC students report experiencing food and housing insecurity. If you are facing challenges in securing food (such as not being able to afford groceries or get sufficient food to eat every day) and housing (such as lacking a safe and stable place to live), please contact the Dean of Students for support (\href{http://studentaffairs.cofc.edu/about/salt.php}). Also, you can go to \href{http://studentaffairs.cofc.edu/student-food-housing-insecurity/index.php} to learn about food and housing assistance that is available to you. In addition, there are several resources on and off campus to help. You can visit the Cougar Pantry in the Stern Center (2nd floor), a student-run food pantry that provides dry-goods and hygiene products at no charge to any student in need. Please also consider reaching out to Professor ABC if you are comfortable in doing so. 


\subsection{Center for Student
Learning}\label{center-for-student-learning}

I encourage you to utilize the Center for Student Learning's (CSL)
academic support services for assistance in study strategies and course
content. They offer tutoring, Supplemental Instruction, study skills
appointments, and workshops. Students of all abilities have become more
successful using these programs throughout their academic career and the
services are available to you at no additional cost. For more
information regarding these services please visit the CSL
\href{http://csl.cofc.edu} or call (843) 953-5635.

\vspace{0.10in}
\noindent I encourage you to take advantage of the Writing Lab in the Center for Student Learning (Addlestone Library, first floor). Trained writing consultants can help with writing for all courses; they offer one-to-one consultations that address everything from brainstorming and developing ideas to crafting strong sentences and documenting sources. For more information, please call 843.953.5635 or visit \href{http://csl.cofc.edu/labs/writing-lab/}.


\subsection{Religious Accommodation for Students}\label{religious-holiday-policy}

It is the policy of the College to excuse absences of students that
result from religious observances and to provide without penalty for the
rescheduling of examinations and additional required class work that may
fall on religious holidays. Please see me immediately if you will need
to miss class any time during this semester.
