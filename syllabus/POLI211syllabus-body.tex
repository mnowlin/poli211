\hypertarget{covid-19}{%
\section{COVID-19}\label{covid-19}}

The COVID-19 pandemic is still ongoing. \textbf{The College of
Charleston requires that masks be worn while indoors and you must wear a
mask at all times while in class.} Although vaccinations are currently
not required, \emph{I ask you to be respectful of the health and safety
of others}. If you have not received the \textbf{COVID-19 vaccine, which
is safe, free, and effective, please consider doing so immediately}.
Information about the vaccine is available from the
\href{https://scdhec.gov/covid19/covid-19-vaccine}{SCDHEC website} and
information about where and when to obtain a vaccine is also available
on the SCDHEC website \href{https://vaxlocator.dhec.sc.gov/}{vaccine
locator page}.

\hypertarget{course-description}{%
\section{Course Description}\label{course-description}}

\begin{quote}
\emph{Our responsibility is one of decision---for to govern is to
choose} - John F. Kennedy
\end{quote}

\vspace{0.1in}

\noindent As stated by President Kennedy, making choices is at the heart
of governing. How do policymakers make choices regarding public policy?
This course will address this question by examining the policymaking
process in the United States.

\vspace{0.1in}

\noindent Throughout this course we will explore the historical and
social context in which policymaking occurs; how problems reach the
agendas of policymakers and how policies are formed to address those
problems; the adoption of specific policy alternatives; how policies are
implemented; and, finally, how public policies are evaluated.

\vspace{0.1in}

\noindent \textbf{Course Catalog}: This course examines the cultural,
economic, and institutional contexts that shape U.S. public policy. The
course examines the processes by which policy problems are addressed and
alternate solutions are adopted. Implications for solving public
problems and resolving political disagreements in a manner consistent
with democratic ideas are considered.

\vspace{0.1in}

\noindent This course will be \emph{lecture} and \emph{discussion}
based. Being able to adequately participate requires you to come to
class prepared by having done the assigned readings prior to class. In
addition, you should be prepared to participate in class by asking
questions and making informed comments that add to the class discussion.
\textbf{I may call on you to answer a question or discuss your policy
issue}.

\vspace{0.1in}

\noindent Laptops are allowed, but discouraged. Phones are only to be
used to answer quiz questions. \emph{I encourage you to take notes by
hand, with pen and paper}.
\href{https://www.nytimes.com/2017/11/27/learning/should-teachers-and-professors-ban-student-use-of-laptops-in-class.html}{You
learn better that way}. I recommend taking notes using the
\href{http://www.usu.edu/arc/idea_sheets/pdf/note_taking_cornell.pdf}{Cornell
Method}. Also, lecture slides will generally \textbf{not} be made
available outside of class.

\hypertarget{course-goals-and-learning-objectives}{%
\subsection{Course Goals and Learning
Objectives}\label{course-goals-and-learning-objectives}}

The goals for this course are to:

\begin{itemize}
\item
  Understand the public policy process in the US
\item
  Apply various public policy models to real world policy issues
\item
  Analyze the merits of public policy debates
\item
  Analyze the merits of alternative policy solutions to public problems
\end{itemize}

\hypertarget{general-social-science-education-learning-outcomes}{%
\subsubsection{General Social Science Education Learning
Outcomes}\label{general-social-science-education-learning-outcomes}}

Upon completion of this course students should be able to apply social
science concepts, models or theories to explain human behavior, social
interactions or social institutions. This will be assessed in the final
exam.

\hypertarget{required-materials}{%
\subsection{Required Materials}\label{required-materials}}

The following materials are \textbf{required}.

\begin{itemize}

\item
  \emph{Readings}:

  \begin{itemize}
  
  \item
    Birkland, Thomas A. 2020. \emph{An Introduction to the Policy
    Process: Theories, Concepts, and Models of Public Policy Making} 5th
    Edition. \emph{4th edition is fine}.
  \item
    Additional readings listed in the schedule will be available on
    \href{https://lms.cofc.edu/d2l/login}{OAKS}.
  \end{itemize}
\item
  \emph{Poll Everywhere}: You are required to set-up an account and
  register your phone with Poll Everywhere.

  \begin{itemize}
  
  \item
    \emph{There is no cost to use Poll Everywhere for this class}
  \item
    I encourage you to review the materials
    \href{https://www.polleverywhere.com/guides/student}{here} and
    \href{https://blog.polleverywhere.com/students-poll-everywhere-101/}{here}
  \end{itemize}
\end{itemize}

\hypertarget{course-prerequisites}{%
\subsection{Course Prerequisites}\label{course-prerequisites}}

POLI 101 or permission of instructor

\hypertarget{attendance-policy}{%
\subsection{Attendance Policy}\label{attendance-policy}}

Attendance will not be taken; however, a lack of attendance will result
in missed quiz questions. Additionally, lecture slides will \emph{not}
be made available outside of class. \textbf{Do not come to class if you
feel ill or if you have been exposed to COVID-19, regardless of how you
feel. I am happy to meet with you to discuss material you missed.}

\hypertarget{course-requirements-and-grading}{%
\subsection{Course Requirements and
Grading}\label{course-requirements-and-grading}}

Performance in this course will be evaluated on the basis of in-class
quiz questions, four issue papers, a policy memo, and two exams. Points
will be distributed as follows:

\vspace{0.1in}
\begin{tabular}{ l l}
\hline
Assignment & Possible Points \\ 
\hline
Quiz Questions & 200 points total \\
Mid-Term Exam & 100 points \\ 
Final Exam & 100 points \\
Issue Papers & (4 at 100 pts each) 400 points total \\
\hline
Total &  800 points \\
\hline
\end{tabular}

\hypertarget{assignments}{%
\subsubsection{Assignments}\label{assignments}}

All due dates for assignments are on the following schedule.

\vspace{0.1in}

\noindent \emph{Quiz Questions}: There will be 1 to 2 quiz questions
given during each class period and the questions will be answered using
Poll Everywhere on your phone. \emph{You must be present in class to be
able to answer the questions}. These questions will cover material from
the readings and/or class discussion. Each question will be worth 5
points and can not be made up if you miss class. However, \textbf{I will
add up to 25 points to your quiz questions grade at the end of the
course.}

\vspace{0.1in}

\noindent \emph{Mid-term}: The mid-term exam will be given on
\textbf{Thursday March 3} and will be \textbf{taken in OAKS}. All
material from the readings, lectures, and in-class discussions are fair
game for the mid-term exam. The exam will be multiple choice, short
answer, and short essay.

\vspace{0.1in}

\noindent \emph{Final Exam}: \textbf{The final exam period is ‌Monday
May 2} from \textbf{1:00 PM to 3:00 PM} and it will also be
\textbf{taken on OAKS}. The final will \emph{NOT} be comprehensive and
all material from the readings, lectures, and in-class discussions
\emph{since the mid-term} are fair game. The exam will be multiple
choice, short answer, and short essay.

\vspace{0.1in}

\noindent \emph{Issue Papers}: You will pick a problem or policy issue
of interest to you and you will write \textbf{four} short, 4 to 5 page,
papers about various aspects of the issue. These aspects include the
nature of the problem; current and past policies to address the problem;
and several alternative approaches to the problem. Details for each
issue paper are posted on OAKS. \textbf{Issue papers must be turned in
through the Assignments folder on OAKS}.

\begin{itemize}

\item
  \textbf{Issue paper 1 due Feb 3}
\item
  \textbf{Issue paper 2 due Feb 24}
\item
  \textbf{Issue paper 3 due March 24}
\item
  \textbf{Issue paper 4 due April 21}
\end{itemize}

\hypertarget{late-work-policy}{%
\paragraph{Late Work Policy}\label{late-work-policy}}

Late work is subject to a 48-hour grace period, and after that will be
penalized 10\% each day (24 hr period) it is late, up to 3 days. After 3
days the assignment will not be accepted. For example, if an assignment
is due Thursday at 2:00 PM, the grace period ends on Saturday at 2:00 PM
and it is late as of 2:01 PM and you lose 10\%. After Sunday at 2:01 PM
you lose another 10\%, after Monday at 2:01 PM another 10\%, and no work
will be accepted after Tuesday at 2:00 PM. \emph{No late work will
accepted 72 hrs after the assignment due date and time}.

\hypertarget{grading-scale}{%
\subsubsection{Grading Scale}\label{grading-scale}}

There are \textbf{800} possible points for this course. Grades will be
allocated based on your earned points and calculated as a percentage of
\textbf{800}. A: 94 to 100\%; A-: 90 to 93\%; B+: 87 to 89\%; B: 83 to
86\%; B-: 80 to 82\%; C+: 77 to 79\%; C: 73 to 76\%; C-: 70 to 72\%; D+:
67 to 69\%; D: 63 to 67\%; D-: 60 to 62\%; F: 59\% and below
