\hypertarget{course-description}{%
\section{Course Description}\label{course-description}}

\emph{Our responsibility is one of decision---for to govern is to
choose. - John F. Kennedy}

\vspace{0.15in}

\noindent As stated by President Kennedy, making choices is at the heart
of governing. How do policymakers make choices regarding public policy?
This course will address this question by examining the policymaking
process in the United States.

\vspace{0.1in}

\noindent Throughout this course we will explore the historical and
social context in which policymaking occurs; how problems reach the
agendas of policymakers and how policies are formed to address those
problems; the adoption of specific policy alternatives; how policies are
implemented; and, finally, how public policies are evaluated.

\hypertarget{course-goals-and-learning-objectives}{%
\subsection{Course Goals and Learning
Objectives}\label{course-goals-and-learning-objectives}}

The goals for this course are to:

\begin{itemize}

\item
  Understand the public policy process in the US
\item
  Apply various public policy models to real world policy issues
\item
  Analyze the merits of public policy debates
\item
  Analyze the merits of alternative policy solutions to public problems
\end{itemize}

\hypertarget{general-social-science-education-learning-outcomes}{%
\subsubsection{General Social Science Education Learning
Outcomes}\label{general-social-science-education-learning-outcomes}}

Upon completion of this course students should be able to apply social
science concepts, models or theories to explain human behavior, social
interactions or social institutions. This will be assessed in the final
exam.

\hypertarget{delivery-format}{%
\subsection{Delivery Format}\label{delivery-format}}

This is an asynchronous online course, and so it is largely self-paced.
Students must have access to a \textbf{computer} with \textbf{high-speed
internet access} throughout the course. In addition, students must have
access to \textbf{OAKS} and \emph{should check OAKS frequently (AT LEAST
every other day) to be sure not to fall behind}. Finally, students must
have access to their \textbf{CofC email}. \textbf{Computer
failure/unavailability does not constitute an excuse for not completing
assignments by the due date}.

\vspace{0.1in}

\noindent It is essential that students stay on top of the course
assignments. I will post due dates, but it is your responsibility to
make sure you don't get behind, especially in a class this short. Do not
make the mistake of thinking this is an easy class because we're meeting
online, or an easy class because it's meeting over the summer. The
material is quite difficult, and will take a lot of effort on your part
to master. \emph{A summer class that meets face-to-face normally entails
three hours of classroom time per weekday, plus reading and homework
each night. The workload for this class will be the same, except our
classroom will be OAKS}.

\hypertarget{technical-issues}{%
\subsubsection{Technical Issues}\label{technical-issues}}

If you have technical problems, please contact the Student Computing
Support Desk at 843.953.8000 or email
\url{studentcomputingsuport@cofc.edu}.

\hypertarget{contacting-the-professor}{%
\subsection{Contacting the Professor}\label{contacting-the-professor}}

If you have questions about course related material, and/or course
procedures please \emph{post your question to the Course Questions
discussion board on} \href{lms.cofc.edu}{OAKS}, so that other students
can benefit from your questions and the answer. I will respond to
discussion board questions within 48 hours, \emph{if not sooner}. If you
are having problems with \emph{course material}, please feel free to
email me at \url{nowlinmc@cofc.edu}.

\hypertarget{email-policy}{%
\subsubsection{Email Policy}\label{email-policy}}

Email is the best way to contact me and I am happy to answer questions
and/or address concerns over email. Please note the following:

\begin{enumerate}
\def\labelenumi{\arabic{enumi}.}

\item
  Please allow 24 hours for a response from me before sending a second
  email.
\item
  Assignments must be turned in through the corresponding Assignment
  folder on OAKS and will not be accepted by email under any
  circumstances, including ``issues with OAKS.''
\item
  If you are having a technical issue with OAKS, I will not be able to
  help you, so please contact the
  \href{https://it.cofc.edu/help/studentcomputing.php}{Student Computing
  Support Desk}.
\end{enumerate}

\hypertarget{required-materials}{%
\section{Required Materials}\label{required-materials}}

\begin{itemize}
\item
  Birkland, Thomas A. 2020. \emph{An Introduction to the Policy Process:
  Theories, Concepts, and Models of Public Policy Making} 5th Edition.
  This book is required and a print version is available at the College
  Bookstore. \emph{Be sure you have the 5th edition}.
\item
  Access to \href{https://lms.cofc.edu/d2l/login}{OAKS}. We will make
  extensive use of OAKS in this course and several of its tools
  including Discussion Boards, Quizzes, and Assignments. Tutorials for
  each of these tools can be found
  \href{http://blogs.cofc.edu/oaks/students/tutorials/}{here}.
\end{itemize}

\hypertarget{navigating-this-course}{%
\section{Navigating This Course}\label{navigating-this-course}}

Course material will be organized into 4 content modules that you will
be able to access on OAKS beginning each \textbf{Tuesday starting on
Tuesday, July 13.} Each module will consist of:

\begin{itemize}
\item
  Readings from the \emph{Birkland} book
\item
  Lectures
\item
  A policy issue that includes readings, a video, and a discussion board
\item
  Module quiz
\end{itemize}

\vspace{0.1in}

\noindent Each module will be made available at 7:00 AM on each Tuesday,
and assignments within each module are due by \emph{11:59 PM on each
Monday}.

\hypertarget{assignments}{%
\subsection{Assignments}\label{assignments}}

Your grade in this course will be determined by your performance on 4
module quizzes, 16 discussion board posts, two exams and a research
presentation. Detailed instructions for each assignment will be
available on OAKS.

\vspace{0.15in}

\begin{tabular}{l | r | r}
\hline
Assignment & Possible Points & \% of Grade \\
\hline
Module quizzes & 100 (total) & 20\%  \\
Policy issue discussion boards & 100 (total) & 20\%  \\
Exam I & 100 & 20\% \\
Exam II & 100 & 20\% \\
Research presentation & 100 &  20\% \\  
\hline
\textbf{Total} & 500 & 100\% \\
\hline
\end{tabular}

\hypertarget{module-quizzes}{%
\subsubsection{Module Quizzes}\label{module-quizzes}}

Each of the 4 modules will have a 10 multiple-choice question quiz over
the readings and lectures for that module. You can use course materials
for the quiz, but you must be take each quiz by yourself. \emph{Once you
begin the quiz you will 10 minutes to complete it}. Each quiz is worth
up to \emph{25 points}.

\hypertarget{discussion-boards}{%
\subsubsection{Discussion Boards}\label{discussion-boards}}

Within each module there will be a discussion board based on a policy
issue. The discussion board will involve a discussion question about a
policy issue that connects to concepts presented in the \emph{Birkland}
readings. To gain an understanding of the various policy issues, you
will be assigned a reading and video. \emph{You should do all the
readings in the module and watch the video before you answer the
discussion question}. For each policy issue discussion board you must a)
provide an \emph{response} \textbf{of about 3 to 4 paragraphs} to the
discussion question and b) \emph{comment} on one other students answer,
\textbf{in about a paragraph}. Note that you will not be able to see or
comment on another student's post until you provide your response to the
question. Each discussion board is worth up to \emph{25 points}, 18.75
points for your response and 6.25 points for your comment.

\vspace{0.1in}

\noindent \emph{You should not make any statement to or about anyone in
an email or discussion board that you would not make in person. Be
respectful of your classmates. In this course we will address policy
issues that are controversial. All discussion of issues will be
respectful to differing views. Finally, this course will be about
learning to approach policy issues as scholars and policy analysts, not
as partisans for one particular point of view.}

\hypertarget{exam-i}{%
\subsubsection{Exam I}\label{exam-i}}

Exam I will be available on OAKS on \textbf{Monday, July 26 at 7:00 AM
and must be completed by 11:59 PM that same day}. The exam will be
\emph{four short essay} questions and cover material from Modules 1 and
2. Each answer should be \textbf{about 500 words}. \emph{You are allowed
to use course material for the exam, however, you must take the exam on
your own}.

\hypertarget{exam-ii}{%
\subsubsection{Exam II}\label{exam-ii}}

Exam II will be available on OAKS starting \textbf{Monday, August 10 at
7:00 AM and must be completed by 11:59 PM that same day}. The exam will
be \emph{four short essay} questions and cover material from Modules 3
and 4. Each answer should be \textbf{about 500 words}. \emph{You are
allowed to use course material for the exam, however, you must take the
exam on your own}.

\hypertarget{research-presentation}{%
\subsubsection{Research Presentation}\label{research-presentation}}

Students will create a presentation that profiles a pressing issue of
concern at the federal, state, or local level. Examples include climate
change, taxes, criminal justice, employment, or education. \textbf{You
may not choose any of the issues used in the course (democracy, police
reform, healthcare, or immigration).} The presentation should include an
overview of the issue, current policies in place to address the issue,
and a proposed policy solution to the issue. \emph{You should send me an
email with your presentation topic by July 23}. Students must create a
PowerPoint with audio or video narration in VoiceThread for this
presentation. Be creative and feel free to include pictures, interviews
of public figures, etc. The presentation should be about 10 minutes and
is \textbf{due by 11:59 PM on Monday, August 9.}

\hypertarget{grades}{%
\subsection{Grades}\label{grades}}

There are \textbf{500} possible points for this course. Grades will be
allocated based on your earned points and calculated as a percentage of
\textbf{500}. A: 94 to 100\%; A-: 90 to 93\%; B+: 87 to 89\%; B: 83 to
86\%; B-: 80 to 82\%; C+: 77 to 79\%; C: 73 to 76\%; C-: 70 to 72\%; D+:
67 to 69\%; D: 63 to 67\%; D-: 60 to 62\%; F: 59\% and below.

\hypertarget{course-content}{%
\section{Course Content}\label{course-content}}

The following modules will be on OAKS and made available on the dates
indicated. Course content and schedule is subject to change. Changes
will be announced through email.

\hypertarget{module-1-studying-policy-and-policymaking-in-context}{%
\subsection{Module 1: Studying Policy and Policymaking in
Context}\label{module-1-studying-policy-and-policymaking-in-context}}

\hypertarget{available-700-am-on-july-13}{%
\paragraph{Available 7:00 AM on July
13}\label{available-700-am-on-july-13}}

\begin{itemize}

\item
  \emph{Readings}:

  \begin{itemize}
  
  \item
    Birkland chapter 1: \emph{Introducing the Policy Process}
  \item
    Birkland chapter 2: \emph{Elements of the Policy Making System}
    (pgs., 32-37; skim the rest)
  \item
    Birkland chapter 3: \emph{The Historical Contexts of Policy Making}
  \item
    Dahl chapter 4: \emph{What is Democracy}? (Available on
    \href{https://lms.cofc.edu/d2l/home}{OAKS})
  \end{itemize}
\item
  \emph{Policy Issue}: \textbf{Democracy}

  \begin{itemize}
  
  \item
    \emph{Reading}: CQ Researcher \emph{Voting Rights}
  \item
    \emph{Video}:
    \href{https://www.pbs.org/wgbh/frontline/film/american-insurrection/}{American
    Insurrection}
  \end{itemize}
\item
  \emph{All assignments due by 11:59 PM on July 19}
\end{itemize}

\hypertarget{module-2-actors-institutions-and-agenda-setting}{%
\subsection{Module 2: Actors, Institutions, and Agenda
Setting}\label{module-2-actors-institutions-and-agenda-setting}}

\hypertarget{available-700-am-on-july-20}{%
\paragraph{Available 7:00 AM on July
20}\label{available-700-am-on-july-20}}

\begin{itemize}

\item
  \emph{Readings}:

  \begin{itemize}
  
  \item
    Birkland chapter 4: \emph{Official Actors and Their Roles in Public
    Policy}
  \item
    Birkland chapter 5: \emph{Unofficial Actors and Their Roles in
    Public Policy}
  \item
    Birkland chapter 6: \emph{Agenda Setting, Groups, and Power}
  \end{itemize}
\item
  \emph{Policy Issue}: \textbf{Healthcare}

  \begin{itemize}
  
  \item
    \emph{Reading}: CQ Researcher \emph{Health and Society}
  \item
    \emph{Video}:
    \href{https://www.pbs.org/video/the-healthcare-divide-rv6npd/}{The
    Healthcare Divide}
  \end{itemize}
\item
  \emph{All assignments due by 11:59pm EST on July 26}
\end{itemize}

\hypertarget{email-research-presentation-topic-by-july-23}{%
\subsubsection{Email research presentation topic by July
23}\label{email-research-presentation-topic-by-july-23}}

\hypertarget{exam-i-july-26}{%
\subsubsection{Exam I: July 26}\label{exam-i-july-26}}

\hypertarget{module-3-policy-types-and-decision-making}{%
\subsection{Module 3: Policy Types and
Decision-Making}\label{module-3-policy-types-and-decision-making}}

\hypertarget{available-700-am-on-july-27}{%
\paragraph{Available 7:00 AM on July
27}\label{available-700-am-on-july-27}}

\begin{itemize}

\item
  \emph{Readings}:

  \begin{itemize}
  
  \item
    Birkland chapter 7: \emph{Policies and Policy Types}
  \item
    Birkland chapter 8: \emph{Decision-Making and Policy Analysis}
  \end{itemize}
\item
  \emph{Policy Issue}: \textbf{Police Reform}

  \begin{itemize}
  
  \item
    \emph{Reading}: CQ Researcher \emph{Police Under Scrutiny}
  \item
    \emph{Video}:
    \href{https://www.pbs.org/video/policing-the-police-2020-ko2tft/}{Policing
    the Police}
  \end{itemize}
\item
  \emph{All assignments due by 11:59 PM on August 2}
\end{itemize}

\hypertarget{module-4-policy-design-and-implementation}{%
\subsection{Module 4: Policy Design and
Implementation}\label{module-4-policy-design-and-implementation}}

\hypertarget{available-700-am-on-august-3}{%
\paragraph{Available 7:00 AM on August
3}\label{available-700-am-on-august-3}}

\begin{itemize}

\item
  \emph{Readings}:

  \begin{itemize}
  
  \item
    Birkland chapter 9: \emph{Policy Design and Policy Tools}\\
  \item
    Birkland chapter 10: \emph{Policy Implementation, Failure, and
    Learning}
  \end{itemize}
\item
  \emph{Policy Issue}: \textbf{Immigration}

  \begin{itemize}
  
  \item
    \emph{Reading}: CQ Researcher \emph{Immigration Overhaul}
  \item
    \emph{Video}:
    \href{https://www.pbs.org/video/zero-tolerance-en2plm/}{Zero
    Tolerance}
  \end{itemize}
\item
  \emph{All assignments due by 11:59 PM on August 9}
\end{itemize}

\hypertarget{research-presentation-due-august-9}{%
\subsubsection{Research Presentation due August
9}\label{research-presentation-due-august-9}}

\hypertarget{exam-ii-august-10}{%
\subsubsection{Exam II: August 10}\label{exam-ii-august-10}}
